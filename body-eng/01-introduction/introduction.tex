\section{Introduction}
\label{sec:intro}

\subsection{Motivation}
\label{sec:intro/motivation}
The ability to create 3D representations of oneself, namely, 3D avatars, has gained the attention of the crowd lately. From the non-research groups that have needs for their self-avatar creation to the researchers who actively work in related fields, it appears that the attention on that is much higher than that of a decade ago \citationneeded. A simple explanation for that is that 3D avatar technology has found its way into practical usage.

First, with the emergence of VR technology, people now want to see others in the virtual worlds more vividly than in non-VR 3D scenarios. That means they want their and others' avatars to express emotions freely, and to be able to represent their personas accurately. Secondly, traditional methods of creating a 3D scene in animation involve manually constructing 3D characters with 3D creation software. That usually costs a lot of money and time, as 3D graphic work requires skills and hundreds of hours to create satisfactory 3D objects. The traditional methods often give better output, but for some people that can be overkill. Moreover, using a lot of money to hire people to produce 3D works can be detrimental to certain companies' financial situation. These two reasons can be why the automated approaches to 3D avatar reconstruction/creation are emerging.

Therefore, I've been researching methods that can simplify or automatically reconstruct 3D avatars from limited input. In the process of researching the best solution to this problem, I found that machine-learning methods can output great results for generative works. With the support of Dr. Ma Thi Chau and the HMI laboratory, I was able to create a system for automated 3D avatar reconstruction and improve it gradually using machine learning methods. The system was then evaluated and brought into use, and achieved great results, which will be elaborated in Chapter \ref{sec:results}).

Thanks to all the supported I've received, especially from Dr. Chau, I was able to present this system in ICTA 2023 - an international conference on Advances in Information and Communication Technology.



% Computer graphics problems are usually interesting to solve because most things we do on the computer are closely related to computer graphics. When I first entered the HMI lab, Dr. Chau presented me with a lot of exciting and practical computer graphics problem, and one field of computer graphics problems is human-machine interactions, where you solve problems about how a machine should react when given a human input. And in that field, I was given a task to research and implement a solution related to 3D avatar reconstruction. <Side note>Reconstruction means that the computer finds a way to create a faithful construction of the input given<end Side note>. 3D avatar reconstruction isn't a simple problem because there are many variables in the output. To put it simply, the output of this problem should be a 3D object that look like the input, which means the 3D object should have the shape of the real-life human head, and the texture - or the skin color of the input human head.

% The process to create the output for the human input can be very limited by many factors. For example, if a machine were to create an 3D avatar from the human image, and if it is given only 1 image, it needs to find a way to compute the head shape. After computing and reconstructing the head shape, it then needs to maps the human skin to the 3D head. However, this is where the hard stuff occurs. First, if the machine is given only one image capturing the human face, then it can only know how to map that front human skin to the 3D head. For the skin part that isn't covered by the image, e.g., the side view or the back view, it doesn't know how to map. So the machine will of course need to GUESS. And guessing the unprovided part is a problem that's highly related to machine learning. So, in order to reconstruct the 3D avatar faithfully, the machine needs to use some machine learning methods to guess the non-provided input, and use algorithmic methods and also machine learning methods to fully reconstruct the 3D model. <Should be in that other section>

% So I researched articles and papers related to 3D avatar reconstruction, and I find that most modern solutions are related to machine learning, and some other stuff. So this paper will explain the machine learning methods that I chose to implement in my system, and other methods (which I came up) needed to create a full fledged 3D avatar reconstruction system.

\subsection{Contributions and thesis overview}
\label{sec:intro/contribution-and-overview}

\subsubsection{Contributions}

The contributions of the thesis involve the creation of the proposed system, which are:

\begin{itemize}
  \item A novel pipeline for handling the 3D reconstruction of avatars from a single-view image, where the hair is created uniquely, separated from the head model.
  \item A method to transfer basic, straightforward human emotions to FLAME - a 3D morphable model's parameters.
\end{itemize}

\subsubsection{Thesis overview}

The rest of this thesis is organized as follows:

Chapter \ref{sec:related-work} provides the related work and fundamentals that are applied to the pipeline of the proposed system.

In chapter \ref{sec:method}, each step of the proposed system's pipeline is explained in detail and with mathematical formulas.

Chapter \ref{sec:results} provides quantitative results of the working system from surveys of the system's users and the experts, and qualitative results in common and specialized metrics.