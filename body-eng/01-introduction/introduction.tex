\section{Introduction}
\label{sec:intro}

\subsection{Motivation}
\label{sec:intro/motivation}
The ability to create 3D representations of oneself, namely, 3D avatars, has gained the attention of the crowd lately. From the non-research groups that have needs for their self-avatar creation to the researchers who actively work in related fields, it appears that the attention on that is much higher than that of a decade ago. A simple explanation for that is that 3D avatar technology has found its way into practical usage.


First, with the emergence of \gls{vr} technology, people now want to see others in the virtual worlds more vividly than in non-\gls{vr} 3D scenarios. That means they want their and others' avatars to express emotions freely, and to be able to represent their personas accurately. Secondly, traditional methods of creating a 3D scene in animation involve manually constructing 3D characters with 3D creation software. That usually costs a lot of money and time, as 3D graphic work requires skills and hundreds of hours to create satisfactory 3D objects. The traditional methods often give better output, but for some people that can be overkill. Moreover, using a lot of money to hire people to create 3D works can be detrimental to certain companies' financial situation. These two reasons can be why the automated approaches to 3D avatar reconstruction/creation are emerging.


Therefore, I've been researching methods that can simplify or automatically reconstruct 3D avatars from limited input. In the process of researching the best solution to this problem, I found that machine-learning methods can output great results for generative works. With the support of Dr. Ma Thi Chau and the \gls{hmi} laboratory, I was able to create a system for automated 3D avatar reconstruction and improve it gradually using machine learning methods. The system was then evaluated and brought into use, and achieved great results (which will be elaborated in Chapter \ref{sec:results}).


Thanks to all the support I've received, especially from Dr. Chau, I was able to present this system in ICTA 2023 - an international conference on Advances in Information and Communication Technology.

\subsection{Contributions and thesis overview}
\label{sec:intro/contribution-and-overview}

\subsubsection{Contributions}

The contributions of the thesis involve the creation of the proposed system, which are:

\begin{itemize}
  \item A novel pipeline for handling the 3D reconstruction of avatars from a single-view image, where the hair is created uniquely, separated from the head model.
  \item A method to transfer basic, straightforward human emotions to FLAME parameters for the ease of customizing facial expressions.
\end{itemize}

\subsubsection{Thesis overview}

The rest of this thesis is organized as follows:

Chapter \ref{sec:related-work} provides the related work and fundamentals that are applied to the pipeline of the proposed system.

In chapter \ref{sec:method}, each step of the proposed system's pipeline is explained in detail and with mathematical formulas.

Chapter \ref{sec:results} provides quantitative results of the working system from surveys of the system's users and the experts, and qualitative results in common and specialized metrics.

Chapter \ref{sec:conclusions} concludes the thesis and provides future work.