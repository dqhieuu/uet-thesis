% FOR PAPER PRINTING
\usepackage[a4paper, top=2.5cm, bottom=3cm, left=3cm, right=2cm]{geometry}
% FOR PDF
% \usepackage[a4paper, margin=2.5cm]{geometry}


% times new roman-like font
\usepackage{newtxtext,newtxmath}

\usepackage[english]{babel}
\usepackage{csquotes}
\setquotestyle{english}

\addto\captionsenglish{
  \renewcommand{\contentsname}%
    {Table of contents}%
  \renewcommand*{\acronymname}{Abbreviations}
}


% allows importing of latex files
\usepackage{subfiles}



% citation
\usepackage[backend=biber,style=ieee]{biblatex}
\addbibresource{ref.bib}

% make conditions like ifstrempty possible
\usepackage{etoolbox}


\usepackage{enumerate}
\renewcommand{\labelenumii}{\theenumii}
\renewcommand{\theenumii}{\theenumi.\arabic{enumii}.}

% Use upright Greek letters as text symbols, e.g. \textbeta
\usepackage{textgreek}

\usepackage{tabularx}

\usepackage{graphicx}
\graphicspath{ {./images/}, {./body/} }


% The drawing library
\usepackage{tikz}
\usetikzlibrary{calc}
\usepackage{eso-pic}


% Wikipedia-style "citation needed" macro
\newcommand{\citationneeded}[1][]{\textsuperscript{\color{blue} [citation needed: #1]}}


\usepackage{hyperref}

% Adds the bibliography, the index, the contents, etc., to the Table of Contents
\usepackage[nottoc,notlot,notlof]{tocbibind}

% Improves the interface for defining floating objects such as figures and tables
\usepackage{float}

% Indent first paragraph after section header
\usepackage{indentfirst}


\usepackage[justification=centering]{caption}

% for table centering
\usepackage{array}


\usepackage{setspace}
\setstretch{1.3}

% Defines commands \counterwithin (which sets up a counter to be reset when another is incremented) and \counterwithout (which unsets such a relationship).
\usepackage{chngcntr}
\counterwithin{figure}{section}
\counterwithin{table}{section}
\setcounter{secnumdepth}{4}
\setcounter{tocdepth}{4}


% Set chapter begin style
\usepackage[explicit]{titlesec}
\titleformat{\section}[display]{\raggedleft\Large}{Chapter \thesection \\
    \vspace{1.5em}
    \bfseries\MakeUppercase{#1}}{0em}{}
\titleformat{name=\section,numberless}{\Large\bfseries\centering\MakeUppercase}{#1}{0em}{}

\titleclass{\subsubsubsection}{straight}[\subsection]
\newcounter{subsubsubsection}[subsubsection]
\renewcommand\thesubsubsubsection{\thesubsubsection.\arabic{subsubsubsection}}
\renewcommand\theparagraph{\thesubsubsubsection.\arabic{paragraph}}

\titleformat{\subsubsubsection}
{\bfseries\itshape}{\thesubsubsubsection \hspace{0.5em} #1}{1em}{}
\titlespacing*{\subsubsubsection}
{0pt}{3.25ex plus 1ex minus .2ex}{1.5ex plus .2ex}

\makeatletter
% \adddialect\l@VIETNAMESE\l@vietnamese

\newcommand\thefontsize[1]{{#1 The current font size is: \f@size pt\par}}

\renewcommand\paragraph{\@startsection{paragraph}{5}{\z@}%
    {3.25ex \@plus1ex \@minus.2ex}%
    {-1em}%
    {\normalfont\normalsize\bfseries}}

\renewcommand\subparagraph{\@startsection{subparagraph}{6}{\parindent}%
    {3.25ex \@plus1ex \@minus .2ex}%
    {-1em}%
    {\normalfont\normalsize\bfseries}}

\def\toclevel@subsubsubsection{4}
\def\toclevel@paragraph{5}
\def\toclevel@paragraph{6}
\def\l@subsubsubsection{\@dottedtocline{4}{7em}{4em}}
\def\l@paragraph{\@dottedtocline{5}{10em}{5em}}
\def\l@subparagraph{\@dottedtocline{6}{14em}{6em}}
\makeatother




% Inline figure
\newcommand{\inlinefigure}[1]{%
    \begingroup\normalfont
    \includegraphics[height=\fontcharht\font`\B]{#1}%
    \endgroup
}


\NewDocumentCommand{\includefigure}{O{} m O{0.9}}{
    \begin{figure}[H]
        \setlength{\abovecaptionskip}{15pt plus 3pt minus 2pt}
        \centering
        \vspace{1cm}
        \captionsetup{font=bf}
        \includegraphics[width=#3\textwidth]{#2}
        \IfValueT{#1}{\caption{#1}}
        \label{fig:#2}
    \end{figure}
}

\NewDocumentCommand{\includesmallfigure}{O{} m}{
    \begin{figure}[H]
        \setlength{\abovecaptionskip}{15pt plus 3pt minus 2pt}
        \centering
        \vspace{1cm}
        \captionsetup{font=bf}
        \includegraphics[width=0.6\textwidth,height=0.6\textheight,keepaspectratio]{#2}
        \IfValueT{#1}{\caption{#1}}
        \label{fig:#2}
    \end{figure}
}

\NewDocumentCommand{\includesvgfigure}{O{} m}{
    \begin{figure}[H]
        \setlength{\abovecaptionskip}{15pt plus 3pt minus 2pt}
        \centering
        \vspace{1cm}
        \includesvg[width=0.9\textwidth,height=0.9\textheight,keepaspectratio]{#2}
        \captionsetup{font=bf}
        \IfValueT{#1}{\caption{#1}}
        \label{fig:#2}
    \end{figure}
}

\usepackage{fancyvrb} % better verbatim environment
\usepackage{catchfile} % fetch file content
\usepackage{plantuml}
\usepackage{xstring} % string manipulation
\usepackage{rotating} % for sideways figures
\usepackage{multirow} % for multirow tables

\NewDocumentEnvironment{umlfigure}{s O{} O{}}{%
    \VerbatimOut{\PlantUMLJobname-plantuml.txt}}
{%
    \endVerbatimOut

    % \makeatletter
    % \CatchFileDef{\UMLCONTENT}{\PlantUMLJobname-plantuml.txt}{\obeylines}
    % \makeatother
    % \IfBeginWith{\UMLCONTENT}{@startuml}{
    % }{
    %     \newwrite\myoutput
    %     \immediate\openout\myoutput=\PlantUMLJobname-plantuml.txt
    %     \immediate\write\myoutput{@startuml^^J\UMLCONTENT^^J@enduml}
    %     \immediate\closeout\myoutput
    % }

    \ifluatex
        \directlua{
            local jobname=\luastring{\PlantUMLJobname}
            local plantUmlMode=\luastring{\PlantUmlMode}
            require("plantuml.lua")
            convertPlantUmlToTikz(jobname, plantUmlMode)
        }
    \else
        \stepcounter{PlantUmlFigureNumber}
        \typeout{*** plantuml only works with lualatex ***}
    \fi
    \ifthenelse{\equal{\PlantUmlMode}{latex}}{
        \IfStrEq{#1}{\BooleanTrue}{
            \begin{sidewaysfigure}
                \centering
                \begin{adjustbox}{max height=\pageheight,max width=0.8\linewidth}
                    \input{\PlantUMLJobname-plantuml.tex}
                \end{adjustbox}
                \captionsetup{font=bf}
                \IfValueT{#2}{\caption{#2}}
                \IfValueT{#3}{\label{fig:#3}}
            \end{sidewaysfigure}
        }{
            \begin{figure}[H]
                \centering
                \begin{adjustbox}{max width=\linewidth}
                    \input{\PlantUMLJobname-plantuml.tex}
                \end{adjustbox}
                \captionsetup{font=bf}
                \IfValueT{#2}{\caption{#2}}
                \IfValueT{#3}{\label{fig:#3}}
            \end{figure}
        }
    }{
        \includegraphics[width=\maxwidth{\textwidth}]{\PlantUMLJobname-plantuml.\PlantUmlMode}
    }
}


\newcommand{\tablelinebreak}[1]{
    \begin{tabular}[c]{@{}l@{}}
        #1
    \end{tabular}
}




% Decorative border
\newcommand{\adddecorativeborder}{
    \AddToShipoutPictureBG*{
        \begin{tikzpicture}[overlay,remember picture]
            \draw [line width=3pt]
            ($ (current page.north west) + (2.5cm,-2.0cm) $)
            rectangle
            ($ (current page.south east) + (-1.5cm,1.8cm) $);
            \draw [line width=1pt]
            ($ (current page.north west) + (2.65cm,-2.15cm) $)
            rectangle
            ($ (current page.south east) + (-1.65cm,1.95cm) $);
        \end{tikzpicture}%
    }
}

\usepackage[automake]{glossaries-extra}
\makeglossaries

\setabbreviationstyle[acronym]{long-short}


% \usepackage{hhline}
% \usepackage{scrextend}

% \usepackage[most]{tcolorbox}
% \definecolor{block-gray}{gray}{0.85}

% \usepackage[inkscapelatex=false]{svg}
% \svgpath{{../images/}}

% \usepackage{varioref} % More descriptive referencing

% \usepackage{pdfpages}

% \usepackage{url}
% \urlstyle{same}
% \def\UrlBreaks{\do\/\do-}
% \usepackage{breakurl}

% \renewcommand{\cleardoublepage}{
%     \clearpage
%     \checkoddpage
%     \ifoddpage
%     \else
%         \hbox{}
%         \vspace*{\fill}
%         \begin{center}
%             \textit{Mặt này được để trống}
%         \end{center}
%         \vspace*{\fill}
%         \thispagestyle{empty}
%         \newpage
%     \fi
% }

% \usepackage{fancyhdr}
% \newcolumntype{P}[1]{>{\centering\arraybackslash}p{#1}}
% \usepackage[dvipsnames]{xcolor}
% \usepackage{listings}

% \usepackage{multicol}

% \newcommand*{\newglslinked}[5]{%
% \newglossaryentry{#1}%
% {%
% name = {#2},%
% first = {\textbf{#2} (#3)},%
% description = {(#3) #4}%
% see = {#5}
% }%
% }

% \newcommand*{\newgls}[3]{%Gg
%     \newglossaryentry{#1}%
%     {%
%         name = {#2},%
%         first = {\textbf{#2}},%
%         description = {#3}%
%     }%
% }

% \usepackage[toc, page]{appendix}
% \renewcommand{\appendixtocname}{Phụ lục}
% \renewcommand{\appendixpagename}{Phụ lục}
% \newcommand{\appfullname}{}

% \usepackage[skip=6pt plus1pt, indent=1cm]{parskip}

% \usepackage[strict]{changepage}

% \hypersetup{
%     colorlinks=true,
%     allcolors=blue, 
%     bookmarks=true
% }

% \lstset{
%     basicstyle=\ttfamily,
%     language=Java,
%     keywordstyle=\color{Blue},
%     tabsize=4,
%     captionpos=b,
%     breaklines=true,
%     showstringspaces=false
% }

% \pagestyle{fancy}
% \lhead{\appfullname}
% \chead{}
% \rhead{}
% \lfoot{}
% \cfoot{\thepage}
% \rfoot{}

% \renewcommand{\arraystretch}{1.5}